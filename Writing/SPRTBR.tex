\documentclass[12pt,letterpaper]{article}
\usepackage{natbib}

%Packages
\usepackage{pdflscape}
\usepackage{fixltx2e}
\usepackage{textcomp}
\usepackage{fullpage}
\usepackage{float}
\usepackage{latexsym}
\usepackage{url}
\usepackage{epsfig}
\usepackage{graphicx}
\usepackage{amssymb}
\usepackage{amsmath}
\usepackage{bm}
\usepackage{array}
\usepackage[version=3]{mhchem}
\usepackage{ifthen}
\usepackage{caption}
\usepackage{hyperref}
\usepackage{amsthm}
\usepackage{amstext}
\usepackage{enumerate}
\usepackage[osf]{mathpazo}
\usepackage{dcolumn}
\usepackage{lineno}
\usepackage{dcolumn}
\newcolumntype{d}[1]{D{.}{.}{#1}}

\pagenumbering{arabic}


%Pagination style and stuff
\linespread{2}
\raggedright
\setlength{\parindent}{0.5in}
\setcounter{secnumdepth}{0} 
\renewcommand{\section}[1]{%
\bigskip
\begin{center}
\begin{Large}
\normalfont\scshape #1
\medskip
\end{Large}
\end{center}}
\renewcommand{\subsection}[1]{%
\bigskip
\begin{center}
\begin{large}
\normalfont\itshape #1
\end{large}
\end{center}}
\renewcommand{\subsubsection}[1]{%
\vspace{2ex}
\noindent
\textit{#1.}---}
\renewcommand{\tableofcontents}{}
%\bibpunct{(}{)}{;}{a}{}{,}

%---------------------------------------------
%
%       START
%
%---------------------------------------------

\begin{document}

%Running head
\begin{flushright}
Version dated: \today
\end{flushright}
\bigskip
\noindent RH: Branch swapping algorithm

\bigskip
\medskip
\begin{center}

\noindent{\Large \bf SPR/TBR.} %TG: Need a title!
\bigskip

\noindent {\normalsize \sc Thomas Guillerme$^1$$^*$, and Martin D. Brazeau$^1$}\\ %TG: Author order can be swapped of course! There's only a finite combination of 2 elements anyway!
\noindent {\small \it 
$^1$Imperial College London, Silwood Park Campus, Department of Life Sciences, Buckhurst Road, Ascot SL5 7PY, United Kingdom.\\}
\end{center}
\medskip
\noindent{*\bf Corresponding author.} \textit{t.guillerme@imperial.ac.uk}\\  %TG: Same as above
\vspace{1in}

%Line numbering
\modulolinenumbers[1]
\linenumbers

%---------------------------------------------
%
%       ABSTRACT
%
%---------------------------------------------

\newpage
\begin{abstract}
blablabla
\end{abstract}

\noindent (Keywords: )\\

\vspace{1.5in}

\newpage 

%---------------------------------------------
%
%       INTRODUCTION
%
%---------------------------------------------

\section{Introduction}
Classic literature will visit redundant trees \citep{felsenstein2004inferring}.

That takes time and introduces statistical biases (e.g. for heuristic searches, some topologies can be visited more than others).

Therefore it's important to implement branch swapping algorithms properly.

Here we propose a method that is equivalent mathematically to the previous ones \citep{felsenstein2004inferring} but that uses a different practical approach that can be more intuitive and allows to visit each topology only once.

We look at SPR as rerooting + branching.

\section{TBR and SPR}

Define both here. Probably reuse the figure from Felsenstein modified.

\section{Improved algorithm}

\subsection{Neighbour Rule}

\subsection{SPR as SRB (Subtree Rerooting and Branching)}

\section{Conclusion}

This way is not different than the classic SPR/TBR but it's a better implementation.

\section{Data availability and reproducibility}
%TG: Probably some link to morphy


\section{Acknowledgments}
European Research Council under the European Union’s Seventh Framework Programme (FP/2007–2013)/ERC Grant Agreement number 311092.


\bibliographystyle{sysbio}
\bibliography{References}

\end{document}
